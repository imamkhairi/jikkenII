\documentclass[a4j, titlepage]{jarticle}

\usepackage[table,xcdraw]{xcolor}
\usepackage[dvipdfmx]{graphicx}
\usepackage{caption}
% \usepackage{subcaption}
\usepackage{listings}
\usepackage{fancybox}
\usepackage{ascmac}
\usepackage{amsmath}
\usepackage{longtable}
\usepackage{subfig}

\definecolor{codegreen}{rgb}{0,0.6,0}
\definecolor{codegray}{rgb}{0.5,0.5,0.5}
\definecolor{codepurple}{rgb}{0.58,0,0.82}
\definecolor{backcolour}{rgb}{0.95,0.95,0.92}

% Define a custom style
\lstdefinestyle{mystyle}{
    backgroundcolor=\color{backcolour},   
    commentstyle=\color{codegreen},
    keywordstyle=\color{magenta},
    numberstyle=\tiny\color{codegray},
    stringstyle=\color{codepurple},
    basicstyle=\ttfamily\footnotesize,
    breakatwhitespace=false,         
    breaklines=true,                 
    captionpos=b,                    
    keepspaces=true,                 
    numbers=left,                    
    numbersep=5pt,                  
    showspaces=false,                
    showstringspaces=false,
    showtabs=false,                  
    tabsize=2,
    frame=single
}

\lstset{style=mystyle}

\begin{document}
  \begin{center}
  \huge 情報工学実験II\par
  \vspace{15mm}
  \huge テーマ03 \par
  \huge グラフ・ネットワークプログラム \par
  \vspace{15mm}
  % \LARGE タイトル \par
  \vspace{20mm}
  \vspace{100mm}
  \Large 令和5年07月06日 \par
  \vspace{15mm}
  \Large イマム カイリ ルビス \par
  \vspace{10mm}
  \Large 学籍番号:214071\par
  \vspace{10mm}
\end{center}
\clearpage

\tableofcontents
\clearpage

\section{概要}
    \subsection{グラフ理論とは}
    数学においてグラフ理論とは、グラフを研究する学問であり、グラフはオブジェクト間の対関係をモデル化するために用いられる数学的構造である。グラフを構成するためには、点(節点またはノードとも呼ばれる)と辺(枝またはエッジとも呼ばれる)が必要である。 \ref{bib:wikigraph}
    % In mathematics, graph theory is the study of graphs, which are mathematical structures used to model pairwise relations between objects. A graph in this context is made up of vertices (also called nodes or points) which are connected by edges (also called links or lines). \ref{bib:wikigraph}
    
    \subsubsection{有向グラフと無向グラフ}
    有向グラフは、辺の方向が決まっている一方向性のグラフである。一方、無向グラフとは、辺が特定の方向を持たず、双方向性を持つグラフである。
    本実験では、使用したグラフはすべて無向グラフである。更に、

    \subsection{スタック} % ini sepertinya mending ambil buku data structure aja
    In computer science, a stack is an abstract data type that serves as a collection of elements, with two main operations:
    Push, which adds an element to the collection, and
    Pop, which removes the most recently added element that was not yet removed.
    The order in which an element added to or removed from a stack is described as last in, first out, referred to by the acronym LIFO.
    But, in this experiment all stack data structure is made using an array with size of number of nodes, to mimic an actual stack function.

    \subsection{キュー}
    In computer science, a queue is a collection of entities that are maintained in a sequence and can be modified by the addition of entities at one end of the sequence and the removal of entities from the other end of the sequence. 
    ini nanti terusin sampai bikin yang ring itu 
    But, in this experiment queue data structure is made using an array with size of number of nodes, to mimic an actual stack function.

    
    \subsection{実行環境}
    本実験で使用される実行環境:
    \begin{screen}
        \begin{itemize}
            \item プロセッサ:AMD Ryzen 5 5600X
            \item メモリー:16.0 GB
            \item OS:Windows 11 Pro
            \item コンパイラ:gcc
        \end{itemize}    
    \end{screen}

\section{問題1:深さ優先検索(DFS)を用いて検索}
    深さ優先探索は、木やグラフのデータ構造を探索するアルゴリズムである。このアルゴリズムは、根(始点)から開始し、バックトラックする前に各エッジに沿って可能な限り探索する。

    指定したエッジに沿ってこれまでに発見されたノードを追跡し、グラフのバックトラックに役立てるために、スタックが必要となる。

    \subsection{深さ優先検索のプログラム}
    % Depth-first search is worked base on below algorithm:
    深さ優先探索は、以下のアルゴリズムに基づいて行われる:
    add start point to start
    \begin{screen}
        \begin{enumerate}
            \item 始点を出発し,番号の若い順に進む位置を調べ,いけるところまで進む.
            \item 行き場所が無い時,行き場所のある地点まで戻り,再びいけるところまで進む.
            \item 行き場所が全てなくなったら終了.
        \end{enumerate}
    \end{screen}


\end{document}