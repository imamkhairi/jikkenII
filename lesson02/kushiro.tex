\documentclass[a4j]{jarticle}
\usepackage{ascmac}

\begin{document}

\title{釧路の町は大丈夫か}
\author{イマム カイリ ルビs} 
\date{\today}

\maketitle


\section{釧路の町危機説}

\begin{itemize}
    \item 基盤産業がない
    \item 漁業の衰退化
    \item 人口激減
\end{itemize}


\section{釧路の町安心説}

\begin{enumerate}
    \item 魚はくる(ハズだ!)\cite{bib1}
    \item 温暖化で気候はベストになる(ハズだ!)
    \item 中国人受けがよい(ハズだ!)
\end{enumerate}


\section{釧路の特徴}
\begin{description}
    \item[釧路湿原]
    湿原は釧路市の北側に広がる.湿原の大半は,北海道川上郡標茶町と阿寒郡鶴居村、釧路郡釧路町にに属する.湿原の中を釧路川が大きく蛇行しながら流れている.\cite{bib2}

    \item[三大夕日]
    釧路湿原を悠々と蛇行する釧路川では水面を赤く染めてキラキラと輝き,その奥の広大な湿原の風景とバックの
\end{description}

\end{document}