\documentclass[a4j,twocolumn]{jarticle}

%% プリアンブル部 %%
\usepackage{fancybox}
\usepackage{ascmac}
\usepackage{amsmath}

% \newcommand{\OLS}{$\hat{\beta}=(X^TX)^{-1}X^Ty$}

\begin{document}

\title{数式演習}
\author{Ferdi Sambo}
\date{\today}

\maketitle
\thispagestyle{empty}

\begin{equation}
x_{10}=a^5+b^{11}+\sqrt{c} 
\end{equation}

\begin{equation}
x=a + b - c \times d \div e
\end{equation}

\begin{equation}
\sqrt[s]{a+b}
\end{equation}

\begin{equation}
m\vec{a}=\vec{F}
\end{equation}

\begin{equation}
y = x^{4n} + 3x^{2n-1} + 2x - 1
\end{equation}

\begin{equation}
\forall x \in A, \exists x > 0
\end{equation}

\begin{equation}
\Delta ABC \equiv \Delta DEF
\end{equation}

\begin{equation}
\Sigma_{i=1}^n % ini jadinya simbol sigma yang kecil
\sum_{i=1}^n i = 1 + 2 + \dots + n = \frac{n(n+1)}{2}
% \dots 
\end{equation}

\begin{equation}
% \vec{a} \cdot \vec{b} = \left| \vec{a} \right| \left| \vec{b} \right| \cos\theta ?
% ini sepertinya salah, panjang abs value nya beda 
\vec{a} \cdot \vec{b} = \mid \vec{a} \mid \mid \vec{b} \mid \cos\theta 
\end{equation}

\begin{equation}
x = \frac{-b \pm \sqrt{b^2 - 4ac}}{2a}
\end{equation}

\begin{equation}
_nC_k = \frac{n!}{k!(n-k)!}
\end{equation}

\begin{equation}
BELUM INI
\end{equation}

\begin{equation}
y = \sqrt{3x}\log x
\end{equation}

\begin{equation}
\sin(\alpha - \beta) = \sin\alpha\cos\beta - \cos\alpha\sin\beta
\end{equation}

\begin{equation}
1 < 2 \leq \dots \leq x \geq 100 > 99
\end{equation}

\begin{equation}
p = \max \{ k \mid p_k \in X, s_k \notin Y, 1 \leq k \leq n \}
\end{equation}

\begin{equation}
\overline{a+b} = \overline{p} \cdot \overline{q}
% can use bar for one character only
\end{equation}

\begin{equation}
p \land \overline{q} = \overline{p \to q}
\end{equation}

\begin{equation}
P \subset Q, P \subseteq Q
\end{equation}

\begin{equation}
(p \Rightarrow q) \land (p \Leftarrow q) \equiv p \Leftrightarrow q
\end{equation}

\begin{equation}
\lim_{x \to \infty}f(x)
\end{equation}

\begin{equation}
P \cap Q = \emptyset, P \cup Q \neq \emptyset
\end{equation}

\begin{equation}
\int_{x=2}^4 \cot x \log x dx
\end{equation}

\begin{equation}
    \begin{pmatrix}
        1 & 2 & 3\\
        3 & 2 & 1\\
        2 & 1 & 3
    \end{pmatrix}
\end{equation}

\begin{equation}
    F(n) = 
    \begin{cases}
        n \times F(n-1) & (n>1)\\
        1 & (n=1)
    \end{cases}
\end{equation}

\begin{equation}
    belum ini
\end{equation}

\end{document}