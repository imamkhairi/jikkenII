\documentclass[a4j, titlepage]{jarticle}

\usepackage[dvipdfmx]{graphicx}
\usepackage{fancybox}
\usepackage{listings}
\usepackage{xcolor}
\usepackage{colortbl}
\usepackage{ascmac}
\usepackage{fancybox}

\lstdefinestyle{mystyle}{
    backgroundcolor=\color{backcolour},   
    commentstyle=\color{codegreen},
    keywordstyle=\color{magenta},
    numberstyle=\tiny\color{codegray},
    stringstyle=\color{codepurple},
    basicstyle=\ttfamily\footnotesize,
    breakatwhitespace=false,         
    breaklines=true,                 
    captionpos=b,                    
    keepspaces=true,                 
    numbers=left,                    
    numbersep=5pt,                  
    showspaces=false,                
    showstringspaces=false,
    showtabs=false,                  
    tabsize=2
}

\lstset{style=mystyle}

\begin{document}
  \begin{center}
  \huge 情報工学実験II\par
  \vspace{15mm}
  \huge テーマ02 \par
  \huge 乱数を用いたプログラム \par
  \vspace{15mm}
  % \LARGE タイトル \par
  \vspace{20mm}
  \vspace{100mm}
  \Large 令和5年07月06日 \par
  \vspace{15mm}
  \Large イマム カイリ ルビス \par
  \vspace{10mm}
  \Large 学籍番号:214071\par
  \vspace{10mm}
\end{center}
\clearpage

\tableofcontents
\clearpage

\section{概要}
  \subsection{乱数とは}
    乱数とは,ある数字の集合からランダムに選ばれた数字のことである.指定された分布内のすべての数値は,ランダムに選択される確率が等しくなります.

  \subsection{C言語の乱数関数}
    本実験のコードはすべてC言語で書かれている.C言語にはすでに乱数関数が用意されているので,本実験ではそれを利用する.

    \texttt{rand()}の関数は\texttt{stdlib.h}ヘッダーに含まれている.しかし、実行するたびに異なる乱数値を得るためには,\texttt{srand(time(NULL))}という関数も利用する必要がある.つまり,\texttt{time.h}ヘッダーファイルもインクルードする必要がある.

  \subsection{確率とは}
  確率とはある事象の確率は,その事象が起こる可能性を示す数値である.その事象が起こる可能性が高ければ高いほど,確率値は高くなる.\cite{wiki}
  
  確率の基本的な計算は次の公式で計算される.
  \begin{equation}
    P(A) = \frac{f}{N}
  \end{equation}
  事象Aが発生する確率を$P(A)$,事象が発生する可能性の数を$f$,可能な結果の合計数$N$.

  \subsection{実行環境}
    本実験で使用される実行環境:
    \begin{screen}
        \begin{itemize}
            \item プロセッサ:AMD Ryzen 5 5600X
            \item メモリー:16.0 GB
            \item OS:Windows 11 Pro
            \item コンパイラ:gcc
        \end{itemize}    
    \end{screen}

\section{課題1:7個のサイコロを同じ出目になる確率}

\end{document}