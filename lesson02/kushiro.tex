\documentclass[a4j, twocolumn]{jarticle}
\usepackage{ascmac}

\begin{document}

\title{釧路の町は大丈夫か}
\author{イマム カイリ ルビs} 
\date{\today}

\maketitle


\section{釧路の町危機説}

\begin{itemize}
    \item 基盤産業がない
    \item 漁業の衰退化
    \item 人口激減
\end{itemize}


\section{釧路の町安心説}

\begin{enumerate}
    \item 魚はくる(ハズだ!)\cite{bib1}
    \item 温暖化で気候はベストになる(ハズだ!)
    \item 中国人受けがよい(ハズだ!)
\end{enumerate}


\section{釧路の特徴}
\begin{description}
    \item[釧路湿原]
    湿原は釧路市の北側に広がる.湿原の大半は,北海道川上郡標茶町と阿寒郡鶴居村、釧路郡釧路町にに属する.湿原の中を釧路川が大きく蛇行しながら流れている.\cite{bib2}

    \item[三大夕日]
    釧路湿原を悠々と蛇行する釧路川では水面を赤く染めてキラキラと輝き,その奥の広大な湿原の風景とバックの雌阿寒岳,雄阿寒岳のすばらしいコントラストが見ることができる.\cite{bib2}

    \item[霧の町]
    道東の太平洋沿岸では春から夏にかけ海霧がよく発生し,地元の人々はこの霧を「ガス」と呼んでいる.海上から流れて来るの霧は,釧路の港や街を包み,夜になると四季像のある幣舞橋周辺はロマンチックな雰囲気に変わる.\cite{bib2}
\end{description}

\section{釧路の大学}
\begin{enumerate}
    % harusnya ini (1) gitu penomerannya
    \item 北海道教育大学釧路校
    \item 釧路公立大学
    \item 釧路短期大学
\end{enumerate}

\section{打開策}
\begin{shadebox}
    \begin{enumerate}
        \item 釧路出身の国民的アイドルユニットKSR946を誕生される.
        \item 巨大原子力発電所を建設し、電気を高値で東京電力に売る.
        \item 無理やり運河とガラス館を作って第二の小樽を目指す.
        \item 〇〇先生が間違ってノーベル賞を受賞する.
    \end{enumerate}
\end{shadebox}
\begin{itembox}{高専生の活用}
    釧路高専卒業生が将来,大企業を立ち上げ地元釧路に凱旋し、多くの雇用をもたらす.町を活気に溢れ好景気がみなぎり,人口が爆発的に増加し、ついには新幹線や国際空港が開発され政令指定都市となるでしょう.
\end{itembox}

\begin{thebibliography}{99}
    \bibitem{bib1}
    魚群喜多蔵,熱い漏れの海, 昆布森出版, 2010.
    \bibitem{bib2}
    釧路の観光HP,http://kankou.city.kushiro.hokkaido.jp
\end{thebibliography}

\end{document}