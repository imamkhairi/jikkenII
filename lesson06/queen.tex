\documentclass[a4j, twocolumn]{jarticle}

\usepackage[dvipdfmx]{graphicx}

\begin{document}

\title{Queen}
\author{イマム カイリ ルビス\thanks{情報工学分野}}
\date{\today}

\maketitle

\section{クイーンについて}
% Queen are a British rock band formed in London in 1970 by Freddie Mercury (lead vocals, piano), Brian May (guitar, vocals) and Roger Taylor (drums, vocals), later joined by John Deacon (bass). Their earliest works were influenced by progressive rock, hard rock and heavy metal, but the band gradually ventured into more conventional and radio-friendly works by incorporating further styles, such as arena rock and pop rock. 

クイーンは、1970年にロンドンでフレディ・マーキュリー(リードボーカル、ピアノ)、ブライアン・メイ(ギター、ボーカル)、ロジャー・テイラー(ドラム、ボーカル)により結成されたイギリスのロックバンドで、後にジョン・ディーコン(ベース)が参加した。初期の作品はプログレッシブ・ロック、ハード・ロック、ヘヴィ・メタルの影響を受けていたが、次第にアリーナ・ロックやポップ・ロックなどのスタイルを取り入れ、よりコンベンショナルな作品へと変化していった。

\section{クイーンの誕生}

% 2. History: Discuss the early days of Queen, including their first album and their rise to fame in the 1970s. You could also explore some of the key events in their history, such as their performance at Live Aid in 1985.

% Before forming Queen, May and Taylor had played together in the band Smile. Mercury was a fan of Smile and encouraged them to experiment with more elaborate stage and recording techniques. He joined in 1970 and suggested the name "Queen".

% クイーンを結成する前、メイとテイラーはスマイルというバンドで一緒に演奏していた。マーキュリーはSmileのファンで、より凝ったステージやレコーディングのテクニックを試すよう彼らに勧めました。彼は1970年に加入し、「クイーン」という名前を提案した。

% The founding members of Queen met in West London during the late 1960s. Guitarist Brian May had formed the group nmamed 1984 the following year with singer Tim Staffell.[1] May left the group in early 1968 to focus on his degree in Physics and Infrared Astronomy at Imperial College.[2] He formed the group Smile with Staffell (now playing bass) and keyboardist Chris Smith.[3] To complete the line-up, May placed an advertisement on a college notice board for a "Mitch Mitchell/Ginger Baker type" drummer; Roger Taylor, a young dental student, auditioned and got the job.[4]

クイーンの結成メンバーは、1960年代後半に西ロンドンで出会った。ギタリストのブライアン・メイは翌年、ボーカルのティム・スタッフェルと共に1984という名前のグループを結成していた[1]。 
メイは1968年初頭、インペリアル・カレッジで物理学と赤外線天文学の学位取得に専念するためにグループを脱退し、スタッフェル(現在はベース)、キーボーディストのクリス・スミスとSmileというグループを結成した。 [3] 
ラインアップを完成するためにメイが大学の掲示板にドラマーを求める広告を出し、若い歯科学生だったロジャー・テイラーがオーディションを受けてこの仕事にありつけた。[4]


In 1970, Staffell quit Smile, feeling his interests in soul and R&B clashed with the group's hard rock sound and being fed up with the lack of success. [11] The remaining members accepted Freddie Bulsara as lead singer, and recruited Taylor's friend Mike Grose as bassist.

1970年、スタッフェルはソウルとR&Bへの興味がグループのハードロックサウンドと衝突すると感じ、成功しないことに嫌気がさしてスマイルを脱退した [11] 。[11]残ったメンバーはフレディ・ブルサラをリードシンガーとして受け入れ、テイラーの友人マイク・グローズをベーシストとして採用した。

In February 1971, John Deacon joined Queen. In addition to being an experienced bassist, his quiet demeanour complemented the band, and he was skilled in electronics.[17]

1971年2月、ジョン・ディーコンがクイーンに加入した。経験豊富なベーシストであることに加え、彼の静かな物腰はバンドを引き立て、エレクトロニクスに長けていた[17]。


\section{アルバム}

Queen has unique sound, including their use of harmonies, complex arrangements, and diverse range of musical genres. Because of that, most of their album sold well worldwide.

クイーンは、ハーモニー、複雑なアレンジ、多様なジャンルの音楽など、ユニークなサウンドを持っています。そのため、彼らのアルバムのほとんどは世界中で売れました。

Table
https://en.wikipedia.org/wiki/Queen_discography


\section{音の強さ}
4. Perhitungan decibel

Sound intensity can be found from the following equation: I=Δp22ρvw. Δp – change in pressure, or amplitudeρ – density of the material the sound is traveling throughvw – speed of observed sound.

https://phys.libretexts.org/Bookshelves/University_Physics/Book%3A_Physics_(Boundless)/16%3A_Sound/16.2%3A_Sound_Intensity_and_Level
rumus ini

However, a rock concert can be very loud no matter the space. On average, rock concert decibel levels are between 90 and 120 dB. 


6. Conclusion: Sum up the significance of Queen and their enduring appeal to fans around the world.
Even after losing Freddie due to his death in 22 November 1991, Queen still have special place in everyone heart.
In 2002, Queen's "Bohemian Rhapsody" was voted "the UK's favourite hit of all time" in a poll conducted by the Guinness World Records British Hit Singles Book.[431]

At the end of 2004, May and Taylor announced that they would reunite and return to touring in 2005 with Paul Rodgers (founder and former lead singer of Free and Bad Company). Brian May's website also stated that Rodgers would be "featured with" Queen as "Queen + Paul Rodgers", not replacing Mercury. Deacon, who was retired, did not participate.[256]

Queen and Paul Rodgers officially split up without animosity on 12 May 2009.[266]

Queen chose Adam Lambert after meeting him on the set of American Idol in 2009. They become Queen + Adam Lambert and still performing until today.

https://en.wikipedia.org/wiki/Queen_(band)
\end{document}