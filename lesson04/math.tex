\documentclass[a4j]{jarticle}

%% プリアンブル部 %%

\begin{document}

\title{数式の表記の仕方}
\author{おじゃる丸\thanks{Eテレ株式会社} \and 野原しんのすけ\thanks{春日部市役所}}

\maketitle
\thispagestyle{empty}

\LaTeX で文章中に数式を記述したいときは$\sqrt{a^{22}-b_1}$のようにする.
% jadi pangkat 22 harus ^{22}
% untuk di bawah _

別行立てで数式を書きたいなら\[ \int_0^\infty \frac{\sin x}{\sqrt{x}}dx = \sqrt{\frac{\pi}{2}} \]のようにすればいい.
% \[\] -> agar eq dalam ini jadi satu baris sendiri 

さらに,各数式に番号をつけたい場合は,equation環境を使う.すると,

\begin{equation} \label{eq1}
\sqrt{a^2-b} 
\end{equation} 

\begin{equation} \label{eq2}
\int_0^\infty \frac{\sin x}{\sqrt{x}}dx = \sqrt{\frac{\pi}{2}} 
\end{equation}
のようになる.

上記の(\ref{eq1})~式は簡単だが,(\ref{eq2})~式は複雑でおじゃる.

\vspace{2zh}

集合は$A=\{\ 2i \times 5 \mid 1 \le i \leq n\}$のように書くとオシャレだ.% \le < , \leq <=
対数も$log_{10} n$は超ダッサ.正しくは$\log_{10} n$と書く.
sin, cos, max, minも見た目は同じだが数式コマンドを使って,$\sin$, $\cos$, $\max$, $\min$と書くべきだ.
文字xをイタリック体にするとき,\textit{x}とする代わりに,数式スタイルを使って$x$と書いた方が楽だ.
でも,数式スタイルだと$I have a pen$は空白がおかしくなるので,この場合は\textit{I have a pen}を使う.

最後に,数式でも文中でもそうだが,半角カンマ`,'や半角ピリオド`.'を使うときは,必ず直後に半角スペースを入れることを習慣にせよ.
He, 185cm. She, 168cmは恰好いいが,He,185cm.She,168cmはダサい.






\end{document}