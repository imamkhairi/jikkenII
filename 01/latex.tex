\documentclass[a4j]{jarticle}

%% プリアンブル部 %%


\begin{document}

\title{サルでわかる\LaTeX 入門}
\author{イマム カイリ ルビス\thanks{釧路工業高等専門学校}\and Ferdi Smabo\thanks{wkkwland}}
\date{\西暦 \today}

\maketitle

\section{\LaTeX とは何か}

\LaTeX は最高級の組版ソフトである. \LaTeX を使えば,数万円のドットプリンタでも数千万円の写植機でも,その能力を最大限に発揮させることができる.

章番号、節番号などを自動的につけることができるし、目次、索引、貢献リストも自動的に作れる。また、脚注も簡単に書ける。

書体は、和文では明朝とゴシック、欧文では Roman, \textbf{Bold}, \textsf{Sans Serif}, \textit{Italic}, \textsl{Slanted}, \textsc{Small Caps}, \texttt{Typewriter}などが使える。

また、findのfi、officeのffi、flowerのfl、shuffleのffl のような合字(ligature)の処理、VAX, TOYOTA のような寄せ(kerning)の処理、ハイフン処理(hypenation)も自動的に行われる。

数式は、なにしろ米国数学会\footnote{American Mathematical Society}の標準組版システム\cite{Rate06}になってるくらいであるから、\LaTeX は他のどんなシステムよりも自由度があり、美しい組版が可能である。たとえば

\[\int_0^\infty \frac{\sin{x}}{\sqrt{x}}dx
	= \sqrt{\frac{\pi}{2}}\]

といった数式が簡単に組版できる。
同じ数式でも本文中では $\int_0^\infty$ のように書体が自動的に変わる。
更に、数式中の空白(アキ)も自動的に決めてくれる。
記号 $a=b$ のアキ、足算 $a+b$ のアキ、符号 $-a$ の後のアキはみな異なる。

\LaTeX の出力は機種に依存しない。
画面、ドットプリンタ、レーザープリンタ、印刷所の写植機でも全く同じ物を出力することができる\cite{HM99}

\LaTeX のようなソフトを使い慣れてしまうと、もう簡単なワープロソフトは使う気になれなくなる(これはちょっと誇大表現だが...).
特に欧文や数式まじりの文章はワープロでは話にならない(けれはほんとうかも)。

\section{\LaTeX の作者}

\subsection{Knuthについて}

\LaTeX の作者 Donald E. Knuthは1938年1月10日、アメリカWinconsin州に生まれた。
1960年Case Institute of Technology を卒業、1963年 California Institute of Technologyで博士号(数学)を取得、同大学の教壇にたつ。
1968年からはStanford大学コンピュータ科学科教授を務める~\cite{W3TEX}.

\subsection{Knuth の功績}

\begin{itemize}
\item Grace Murray Hopper賞(1971年:ACM)

\item Alan Turing賞(1974年:ACM)

\item Lester R. Ford賞(1975年:MAA)

\item National Medal of Science賞(1979年:USA)

\item McDowell賞(1980年:IEEE)

\item Computer Pioneer賞(1982年:IEEE)
\end{itemize}

\begin{thebibliography}{99}

\bibitem{Rate06}
羅手不二子, LATEXとオープンオフィスは寄生虫, KY出版, 2006.

\bibitem{HM99}
A. Hanage and K. Mimige, "Study on Latex Junkie", J.IEEE, no.4m pp.12--22, 1999.

\bibitem{W3TEX}
日本語TEX情報, "http://oku.edu.mie-u.ac.jp/~okumura/texfaq/".

\end{thebibliography}

\end{document}