\documentclass[a4j, titlepage]{jarticle}

\usepackage[table,xcdraw]{xcolor}
\usepackage[dvipdfmx]{graphicx}
\usepackage{caption}
\usepackage{subcaption}
\usepackage{listings}
\usepackage{fancybox}
\usepackage{ascmac}
\usepackage{amsmath}
\usepackage{longtable}
\usepackage{subfig}

\definecolor{codegreen}{rgb}{0,0.6,0}
\definecolor{codegray}{rgb}{0.5,0.5,0.5}
\definecolor{codepurple}{rgb}{0.58,0,0.82}
\definecolor{backcolour}{rgb}{0.95,0.95,0.92}

% Define a custom style
\lstdefinestyle{mystyle}{
    backgroundcolor=\color{backcolour},   
    commentstyle=\color{codegreen},
    keywordstyle=\color{magenta},
    numberstyle=\tiny\color{codegray},
    stringstyle=\color{codepurple},
    basicstyle=\ttfamily\footnotesize,
    breakatwhitespace=false,         
    breaklines=true,                 
    captionpos=b,                    
    keepspaces=true,                 
    numbers=left,                    
    numbersep=5pt,                  
    showspaces=false,                
    showstringspaces=false,
    showtabs=false,                  
    tabsize=2,
    frame=single
}

\lstset{style=mystyle}

\begin{document}
  \begin{center}
  \huge 情報工学実験II\par
  \vspace{15mm}
  \huge テーマ03 \par
  \huge グラフ・ネットワークプログラム \par
  \vspace{15mm}
  % \LARGE タイトル \par
  \vspace{20mm}
  \vspace{100mm}
  \Large 令和5年07月06日 \par
  \vspace{15mm}
  \Large イマム カイリ ルビス \par
  \vspace{10mm}
  \Large 学籍番号:214071\par
  \vspace{10mm}
\end{center}
\clearpage

\tableofcontents
\clearpage

\section{概要}
    \subsection{グラフ理論とは}
        数学においてグラフ理論とは、グラフを研究する学問であり、グラフはオブジェクト間の対関係をモデル化するために用いられる数学的構造である。グラフを構成するためには、点(節点またはノードとも呼ばれる)と辺(枝またはエッジとも呼ばれる)が必要である\cite{bib:wikigraph}。 
        % In mathematics, graph theory is the study of graphs, which are mathematical structures used to model pairwise relations between objects. A graph in this context is made up of vertices (also called nodes or points) which are connected by edges (also called links or lines). \ref{bib:wikigraph}
        グラフ理論には、辺が方向を持っているかどうかによって分れている
        \begin{description}
            \item[有向グラフ]:辺の方向が決まっている一方向性のグラフである。
            \item[無向グラフ]:辺が特定の方向を持たず、双方向性を持つグラフである。
        \end{description}
        本実験では、使用したグラフはすべて無向グラフである。更に、

    \subsection{スタックとは} 
        スタックは、データを一時的にた蓄えるためのデータ構造の一つ。データの出し入れは\textbf{後入れ先出し}(\textit{LIFO / Last In First Out})で行われる。すなわち、最後に入れられたデータが最初に取り出される\cite{bib:boyoh}。

        なお、スタックにデータを入れる操作を\textbf{プッシュ}(\textit{push})と呼び、スタックからデータを取り出す操作を\textbf{ポップ}(\textit{pop})と呼びます\cite{bib:boyoh}。% ini jadikan ga ada sitasi aja kali

        しかし本実験では、実際のスタック機能を模倣するため、ノード数分の大きさを持つ配列を使ってスタックデータ構造を作った。

    \subsection{キューとは}
        キューは、データを一時的に蓄えるための基本的なデータ構造の一つである。最初に入れられたデータが最初に取り出されるという\textbf{先入れ先出し}(\textit{FIFO / First In First Out})の機構である。\cite{bib:boyoh}
        
        なお、キューにデータを追加する操作を\textbf{エンキュー}(\textit{enqueue})と呼び、データを取り出す操作を\textbf{デキュー}(\textit{dequeue})と呼ぶ。また、データが取り出される側を\textbf{先頭}(\textit{front})と呼び、データが押し込まれる側を\textbf{末尾}(\textit{rear})と呼ぶ\cite{bib:boyoh}。

        しかし本実験では、実際のキュー機能を模倣するため、ノード数分の大きさを持つ配列を使ってスタックデータ構造を作った。

        \subsubsection{リングバッファによるキュー}
            リングバッファとは、配列の末尾が先頭につながっているとみなすデータ構造である\cite{bib:boyoh}。エンキューとデキューを行うと\textit{front}と\textit{rear}の値は変化する。
            % kasih gambar ring buffer


    
    \subsection{実行環境}
    本実験で使用される実行環境:
    \begin{screen}
        \begin{itemize}
            \item プロセッサ:AMD Ryzen 5 5600X
            \item メモリー:16.0 GB
            \item OS:Windows 11 Pro
            \item コンパイラ:gcc
        \end{itemize}    
    \end{screen}

\section{深さ優先検索と幅優先検索を用いて検索}
    \subsection{深さ優先検索}
        深さ優先探索は、木やグラフのデータ構造を探索するアルゴリズムである。このアルゴリズムは、根(始点)から開始し、バックトラックする前に各辺に沿って可能な限り探索する。

        指定した辺に沿ってこれまでに発見されたノードを追跡し、グラフのバックトラックに役立てるために、スタックが必要となる。

        \subsubsection{深さ優先検索のプログラム}
            以下は深さ優先検索のプログラムである.
            \lstinputlisting[language=c]{D:/Kosen/jikkenII/netgraph/dfs_connect.c} 
        
        \subsubsection{深さ優先検索のプログラムの動作}
            訪問したすべての点はスタックにプッシュされ、その点から先に行けない場合はスタックトップがポップされる。
            深さ優先検索のプログラムの主な流れは,以下の通りである:
            \begin{screen}
                \begin{enumerate}
                    \item 根をスタックにプッシュする。
                    \item スタックトップのデータを現在点になるようにピークする。
                    \item 現在の点に接続している最も低い点に進む。
                    \item 全ての点を訪れるまで繰り返す。
                \end{enumerate}
            \end{screen}
        
    \subsection{幅優先検索}
        幅優先探索は、木やグラフのデータ構造を探索するアルゴリズムである。このアルゴリズムは、根(始点)から開始し、各点に隣接している点を訪問する。

        キューは、訪問されたがまだ探索されていない子ノードを追跡するために必要である。

        \subsubsection{幅優先検索のプログラム}
            以下は深さ優先検索のプログラムである.
            \lstinputlisting[language=c]{D:/Kosen/jikkenII/netgraph/bfs_connect.c} 
            


\end{document}